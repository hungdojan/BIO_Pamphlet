\documentclass[a4paper,10pt,twocolumn]{article}
\usepackage[czech]{babel}
\usepackage[left=2cm,text={17cm,24cm},top=3cm]{geometry}
\usepackage[hidelinks]{hyperref}
\usepackage[utf8]{inputenc}
\usepackage[T1]{fontenc}
\usepackage{listings}
\usepackage{xcolor}
\usepackage[perpage]{footmisc}
\usepackage{graphicx}
\usepackage{multirow}
\usepackage{makecell}

\graphicspath{ {./resources/} }

\definecolor{gray}{rgb}{0.95,0.95,0.95}
\definecolor{darkred}{rgb}{0.545,0,0}

\title{Identifikace osoby na základě sítnice oka}

\author{Hung Do, Vojtěch Orava}

\addto\extrasczech{%
  \renewcommand{\sectionautorefname}{Sekce}
  \renewcommand{\subsectionautorefname}{Podsekce}
  \renewcommand{\subsubsectionautorefname}{Podpodsekce}
  \renewcommand{\paragraphautorefname}{Odstavec}
  \renewcommand{\subparagraphautorefname}{Pododstavec}
  \renewcommand{\figureautorefname}{obrázek}
  \renewcommand{\tableautorefname}{Tabulka}
  \renewcommand{\equationautorefname}{Rovnice}
  \renewcommand{\itemautorefname}{Položka}
  \renewcommand{\pageautorefname}{Strana}
}


\begin{document}
    \maketitle
    \thispagestyle{empty}
    \section{Úvod}
    Biometrie je vědní obor zabývajícím se statistickými měřeními a analýzou biologických rysů. 
    V oboru informatiky se tento obor zabývá identifikací a verifikací osob na základě charakteristik a chování daných jedinců.
    Mezi nejznámější metody patří rozpoznávání pomocí \textbf{otisku prstů}, nebo \textbf{obličejů}. 

    Náš projekt se zabýval návrhem a implementací uživatelské aplikace sloužící k identifikaci osob na základě sítnice oka.
    Samotné rozpoznávání je založené na třech odlišných metod: podobnost \textbf{bodů křížení cév}, \textbf{okruží} a \textbf{embedding slepých skvrn}.

    \section{Návrh a implementace jednotlivých metod}
    Zadáním projektu byla implementace identifikace sítnice. Uživatel nahraje obrázek sítnice, která se porovná sítnicemi uložené v databázi.
    Poté, podle vybrané metody, spočítá a vrátí sítnici s nejvýšší mírou shody.

    \subsection{Detekce podle okruží slepé skvrny}
    Každý obrázek sítnice se přetransformuje to čtvercového rozměru 600x600 pixelů. Následně algoritmus vyhledá barevné intenzity v obrazu detekuje slepou skvrnu.
    Slepá skvrna se odřízne a pomocí filtračních operací (metoda Frangi) se segmentují cévy ve slepé skvrně (bitová maska viz. \autoref{fig:blindspot}. Algoritmus vypočítá tloušťku a úhly cév na okruží slepé skvrny.
    Celkem je zpracováno 36 úhlů a vzniklý vektor se uloží do databáze pro pozdější porovnání.
    
    \subsection{Detekce podle bodů křížení cév}
    \label{ssec:crossing}
    Tento algoritmus pracuje s předtrénovaným modelem, který z přijatého obrázku vysegmentuje bitovou masku cév. 
    Algoritmus nejprve předzpracuje obrázek (změní rozměry a zvýší kontrast), poté z obrázku detekuje optický disk a bitovou masku cévní struktury.
    Informace o poloze a rozměrech optického disku slouží pro zarovnání dvou obrázků sítnice. Samotný výpočet je též založený na barevné intenzite, nebo
    koncentrace křížových bodů. Mezi křížové body se řadí cévní překryvy, nebo větvení. Do databáze se nakonec ukládá vektor s informacemi o optické disku (střed a poloměr) a seznam křížových bodů.

    Při porovnání (viz. \autoref{fig:matching}) dvou sítnic se nejprve body zarovnají podle středu optického disku a poté se určí počet bodů, které jsou si podobné na obou obrázcích (pomocí stromové datové struktury \emph{KDTree}). 
    \begin{figure}
        \begin{center}
            \includegraphics[width=0.95\columnwidth]{matching.png}
        \end{center}
        \caption{Zarovnání a porovnání křížových bodů 2 fotografií sítnic stejného oka.}
        \label{fig:matching}
    \end{figure}
    

    \subsection{Detekce pomocí embedding slepé skvrny}
    Metoda nejprve obrázek transformuje do čtvercových rozměrů 600x600 pixelů, a poté odřízne oblast optického disku. 
    Za pomoci předtrénovaného modelu je obraz zakódován do embeddingu (vektor s hodnotami vystihující vlastnosti obrázku). Získaný embedding je uložen do databáze pro pozdější porovnání.

    \begin{figure*}[ht]
        \begin{center}
            \includegraphics[width=0.70\textwidth]{bsn.png}
        \end{center}
        \caption{Porovnání dvou bitových masek cév kolem okruží optického disku.}\label{fig:blindspot}
    \end{figure*}

    \section{Výsledky testování}
    Efektivnost algoritmů se testovalo na menší databázi o šesti sítnicích. Implementovali jsme jednoduché uživatelské prostředí, kde uživatel nejprve inicializuje databázi a poté nahrává samotné obrázky sítnic pro identifikaci.
    Výstupem aplikace jsou informace o dvou sítnicích s nejvyšší pravděpodobností shody (viz. \autoref{fig:gui}). Pro vyhodnocování metod byly vybrány \emph{kosinova podobnost} (hodnoty se pohybují od -1 až 1; čím blíže se hodnota blíží k 1, tím vyšší je podobnost sítnic) a \emph{počet shodných bodů}.

    U první metody (detekce podle okruží slepé skvrny) byla průměrná hodnota kosinovy podobnosti u správně identifikovaných sítnic 0.956. 
    Ukázalo se, že model vrací vysokou podobnost (hodnoty kolem 0,70, nejvyšší dokonce 0,94) i pro sítnice jiných osob a to z důvodu špatně detekovaných optických disků.
    V některých případech metoda detekovala a vyhodnocovala okruží odlesků z blesku při focení místo optického disku.

    Druhá metoda zabývající se křížení cév, byly použity dva různé metody vyhledání pozice optického disku (rozebráno v \autoref{ssec:crossing}).
    Průměrný počet shodných bodů činilo kolem 63 \% vůči všech detekovaných bodů jedné ze sítnic.
    To je zapříčeno z důvodu špatného zarovnání obou sítnic a neodstranění bodů křížení, které se nenacházejí v průniku obou sítnic.
    Nicméně je nutno podotknout, že při přesné zarovnání podle optického disku, bylo u první predikované sítnice nalezeno o 50 \% více shodných bodů, než na druhém místě.

    Poslední metoda (embedding slepé skvrny), podobně jako u té první, byla vyhodnocována za pomoci kosinové podobnosti. 
    U této metody byla průměrná hodnota kosinovy podobnosti 0,929. V jednom ze šesti případů byl uživatel identifikován špatně, kde podobnost predikcí první a druhé sítnice byly velmi podobné.
    Pokudbychom uvažovali pouze správné detekce, průměrná hodnota kosinovy podobnosti by byla 0,983.
    Podobně jako u ostatních metod, i v tomto případě byla chybná identifikace zaviněna
    nesprávnou detekcí optického disku. Výsledky jsou zobrazeny
    v \autoref{tab:resultCosineSimilarity}
    a \autoref{tab:resultCrossing}.

    \begin{table}[t]
      \centering
      \catcode`\-=12
      \resizebox{\columnwidth}{!}{
        \begin{tabular}{|c|c|c|c|c|}
          \hline
          \multirow{2}{*}{\textbf{Sítnice}} & \multicolumn{2}{c|}{\textbf{Okruží slepé
          skvrny}}
          & \multicolumn{2}{c|}{\textbf{Embedding slepé skvrny}} \\ 
          \cline{2-5}
          & \textbf{Podobnost} & \textbf{Správnost} & \textbf{Podobnost}
          & \textbf{Správnost} \\
          \hline
          Osoba 1 & 0,845 & Ano & 0,984 & Ano \\ \hline
          Osoba 2 & 0,993 & Ano & ---   & Ne  \\ \hline
          Osoba 3 & 0,995 & Ano & 0,988 & Ano \\ \hline
          Osoba 4 & 0,997 & Ano & 0,979 & Ano \\ \hline
          Osoba 5 & 0,994 & Ano & 0,984 & Ano \\ \hline
          Osoba A & 0,959 & Ano & 0,974 & Ano \\
          \hline
        \end{tabular}
      }
      \caption{Porovnání metod detekce podle okruží a embeddingu slepých skvrn. Druhá
      metoda nedokázala správně detekovat druhou osobu a jelikož predikovaná osoba nebyla
      mezi prvními dvěma nejlepšími výsledky, hodnotu jsme ponechali prázdnou.}
      \label{tab:resultCosineSimilarity}
    \end{table}

    \begin{table}[t]
      \centering
      \catcode`\-=12
      \resizebox{\columnwidth}{!}{
        \begin{tabular}{|c|c|c|c|c|}
          \hline
          \multirow{2}{*}{\textbf{Sítnice}} & \multicolumn{2}{c|}{\textbf{Metoda prahování}}
          & \multicolumn{2}{c|}{\textbf{Metoda koncentrace bodů}} \\ 
          \cline{2-5}
          & \makecell{\textbf{Počet shod} \\ \textbf{(podíl)}} & \textbf{Správnost}
          & \makecell{\textbf{Početshod} \\ \textbf{(podíl)}} & \textbf{Správnost} \\
          \hline
          Osoba 1 & 44 (81,5 \%) & Ano & 39 (83 \%)   & Ano \\ \hline
          Osoba 2 & 22 (30,1 \%) & Ano & 11 (16,9 \%) & Ne  \\ \hline
          Osoba 3 & 49 (89,1 \%) & Ano & 38 (80,9 \%) & Ano \\ \hline
          Osoba 4 & 55 (82,1 \%) & Ano & 46 (76,7 \%) & Ano \\ \hline
          Osoba 5 & 36 (76,6 \%) & Ano & 29 (74,4 \%) & Ano \\ \hline
          Osoba A & 30 (50,8 \%) & Ano & 28 (48,4 \%) & Ano \\
          \hline
        \end{tabular}
      }
      \caption{Porovnání metod detekce podle okruží a embeddingu slepých skvrn. Hodnoty
      ukazují počet shodných bodů a procentuální shoda vůči všech bodů křížení.}
      \label{tab:resultCrossing}
    \end{table}

    \section{Závěr}
    V tomto projektu jsme vyzkoušeli různé metody identifikace osob za pomocí obrázků sítnice. Všechny metody si předzpracováváli obrázek pro zkvalitnění vlastností, se kterými metody pracovaly.
    Z výsledků testování jsme došli k názoru, že naše metody jsou velmi náchylné na kvalitě vstupního obrazu.
    V případě přítomností odlesků nebo tmavých fotek, měly všechny naše algoritmy problém vyhledat pozici optického disku. Nicméně, při optimálních podmínek všechny metody správně identifikovali sítnice, mnohdy i s vysokou mírou jistoty.

    \begin{figure*}[ht]
        \begin{center}
            \includegraphics[width=0.95\textwidth]{gui.png}
        \end{center}
        \caption{Vzhled grafického uživatelského rozhraní.}\label{fig:gui}
    \end{figure*}
    

\end{document}
